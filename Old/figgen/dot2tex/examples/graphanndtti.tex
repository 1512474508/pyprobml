\documentclass{article}
\usepackage{tikz}
\usetikzlibrary{arrows,shapes}
\usepackage{dot2texi}
\begin{document}
% Define layers
\pgfdeclarelayer{background}
\pgfdeclarelayer{foreground}
\pgfsetlayers{background,main,foreground}

% The scale option is useful for adjusting spacing between nodes.
% Note that this works best when straight lines are used to connect
% the nodes.
\begin{tikzpicture}[>=latex',scale=0.8]
    % set node style
    \tikzstyle{n} = [draw,shape=circle,minimum size=2em,
                        inner sep=0pt,fill=red!20]
    \begin{dot2tex}[dot,tikz,codeonly,styleonly,options=-s -tmath]
        digraph G  {
            node [style="n"];
            A_1 -> B_1; A_1 -> B_2; A_1 -> B_3;
            B_1 -> C_1; B_1 -> C_2;
            B_2 -> C_2; B_2 -> C_3;
            B_3 -> C_3; B_3 -> C_4;
        }
    \end{dot2tex}
    % annotations
    \node[left=1em] at (C_1.west)  (l3) {Level 3};
    \node at (l3 |- B_1) (l2){Level 2};
    \node at (l3 |- A_1) (l1) {Level 1};
    % Draw lines to separate the levels. First we need to calculate
    % where the middle is.
    \path (l3) -- coordinate (l32) (l2) -- coordinate (l21) (l1);
    \draw[dashed] (C_1 |- l32) -- (l32 -| C_4);
    \draw[dashed] (C_1 |- l21) -- (l21 -| C_4);
    \draw[<->,red] (A_1) to[out=-120,in=90] (C_2);
    % Highlight the A_1 -> B_1 -> C_2 path. Use layers to draw
    % behind everything.
    \begin{pgfonlayer}{background}
        \draw[rounded corners=2em,line width=3em,blue!20,cap=round]
                (A_1.center) -- (B_1.west) -- (C_2.center);
    \end{pgfonlayer}
\end{tikzpicture}
\end{document}